%%%%%%%%%%%%%%%%%
% This is an sample CV template created using altacv.cls
% (v1.7.2, 28 Aug 2024) written by LianTze Lim (liantze@gmail.com), based on the
% CV created by BusinessInsider at http://www.businessinsider.my/a-sample-resume-for-marissa-mayer-2016-7/?r=US&IR=T
%
%% It may be distributed and/or modified under the
%% conditions of the LaTeX Project Public License, either version 1.3
%% of this license or (at your option) any later version.
%% The latest version of this license is in
%%    http://www.latex-project.org/lppl.txt
%% and version 1.3 or later is part of all distributions of LaTeX
%% version 2003/12/01 or later.
%%%%%%%%%%%%%%%%

%% Use the "normalphoto" option if you want a normal photo instead of cropped to a circle
% \documentclass[10pt,a4paper,normalphoto]{altacv}

\documentclass[10pt,a4paper,ragged2e,withhyper]{altacv}
%% AltaCV uses the fontawesome5 and simpleicons packages.
%% See http://texdoc.net/pkg/fontawesome5 and http://texdoc.net/pkg/simpleicons for full list of symbols.

% Change the page layout if you need to
\geometry{left=1.25cm,right=1.25cm,top=1.2cm,bottom=1.2cm,columnsep=1.2cm}

% The paracol package lets you typeset columns of text in parallel
\usepackage{paracol}


% Change the font if you want to, depending on whether
% you're using pdflatex or xelatex/lualatex
% WHEN COMPILING WITH XELATEX PLEASE USE
% xelatex -shell-escape -output-driver="xdvipdfmx -z 0" mmayer.tex
\ifxetexorluatex
  % If using xelatex or lualatex:
  \setmainfont{Lato}
\else
  % If using pdflatex:
  \usepackage[default]{lato}
\fi

% Change the colours if you want to
%\definecolor{VividPurple}{HTML}{3E0097}
%\definecolor{SlateGrey}{HTML}{2E2E2E}
%\definecolor{LightGrey}{HTML}{666666}
\definecolor{VividPurple}{HTML}{008080}
\definecolor{SlateGrey}{HTML}{2E2E2E}
\definecolor{LightGrey}{HTML}{666666}
% \colorlet{name}{black}
% \colorlet{tagline}{PastelRed}
\colorlet{heading}{VividPurple}
\colorlet{headingrule}{VividPurple}
% \colorlet{subheading}{PastelRed}
\colorlet{accent}{VividPurple}
\colorlet{emphasis}{SlateGrey}
\colorlet{body}{LightGrey}

% Change some fonts, if necessary
% \renewcommand{\namefont}{\Huge\rmfamily\bfseries}
% \renewcommand{\personalinfofont}{\footnotesize}
% \renewcommand{\cvsectionfont}{\LARGE\rmfamily\bfseries}
% \renewcommand{\cvsubsectionfont}{\large\bfseries}

% Change the bullets for itemize and rating marker
% for \cvskill if you want to
\renewcommand{\cvItemMarker}{{\small\textbullet}}
\renewcommand{\cvRatingMarker}{\faCircle}
% ...and the markers for the date/location for \cvevent
% \renewcommand{\cvDateMarker}{\faCalendar*[regular]}
% \renewcommand{\cvLocationMarker}{\faMapMarker*}


% If your CV/résumé is in a language other than English,
% then you probably want to change these so that when you
% copy-paste from the PDF or run pdftotext, the location
% and date marker icons for \cvevent will paste as correct
% translations. For example Spanish:
% \renewcommand{\locationname}{Ubicación}
% \renewcommand{\datename}{Fecha}


%% Use (and optionally edit if necessary) this .tex if you
%% want to use an author-year reference style like APA(6)
%% for your publication list
% \input{pubs-authoryear.tex}

%% Use (and optionally edit if necessary) this .tex if you
%% want an originally numerical reference style like IEEE
%% for your publication list
%\input{pubs-num.tex}

%% sample.bib contains your publications
%\addbibresource{sample.bib}

\begin{document}
\name{Stéphane Benchimol}
\tagline{Senior Software Engineer \& Devops}
% Cropped to square from https://en.wikipedia.org/wiki/Marissa_Mayer#/media/File:Marissa_Mayer_May_2014_(cropped).jpg, CC-BY 2.0
%% You can add multiple photos on the left or right
\photoR{2,5cm}{cv/me_center.png}
% \photoL{2cm}{Yacht_High,Suitcase_High}
\personalinfo{%
  % Not all of these are required!
  % You can add your own with \printinfo{symbol}{detail}
  \email{stephane.benchimol@gmail.com}
  \phone{+33 6 60 64 50 17}
  \mailaddress{Toulouse, France}
  %\location{Sunnyvale, CA}
  %\homepage{marissamayr.tumblr.com}
  % \twitter{@marissamayer}
  %\xtwitter{@marissamayer}
  %\linkedin{https://www.linkedin.com/in/stephane-benchimol-89378387/}
  \github{steph-ben} % I'm just making this up though.
%   \orcid{0000-0000-0000-0000} % Obviously making this up too.
  %% You can add your own arbitrary detail with
  %% \printinfo{symbol}{detail}[optional hyperlink prefix]
  % \printinfo{\faPaw}{Hey ho!}
  %% Or you can declare your own field with
  %% \NewInfoFiled{fieldname}{symbol}[optional hyperlink prefix] and use it:
  % \NewInfoField{gitlab}{\faGitlab}[https://gitlab.com/]
  % \gitlab{your_id}
	%%
  %% For services and platforms like Mastodon where there isn't a
  %% straightforward relation between the user ID/nickname and the hyperlink,
  %% you can use \printinfo directly e.g.
  % \printinfo{\faMastodon}{@username@instace}[https://instance.url/@username]
  %% But if you absolutely want to create new dedicated info fields for
  %% such platforms, then use \NewInfoField* with a star:
  % \NewInfoField*{mastodon}{\faMastodon}
  %% then you can use \mastodon, with TWO arguments where the 2nd argument is
  %% the full hyperlink.
  % \mastodon{@username@instance}{https://instance.url/@username}
}

\makecvheader

\divider

\textbf{\textit{
With 15 years of experience, I have developed strong expertise in technical coordination, data processing and deploying operational systems for clients. I am now seeking exciting new challenges to apply my skills to innovative projects.
}}
\medskip

%% Depending on your tastes, you may want to make fonts of itemize environments slightly smaller
\AtBeginEnvironment{itemize}{\small}

%% Set the left/right column width ratio to 6:4.
\columnratio{0.6}

% Start a 2-column paracol. Both the left and right columns will automatically
% break across pages if things get too long.
\begin{paracol}{2}

\cvsection{Profesionnal experiences}

\cvevent{Meteo France International}{}{2008 - Today}{Toulouse, France}

\cvevent{}{Senior Software Engineer and Architect}{}{}
\begin{itemize}
\item \textbf{Design solution} for long term capacity storage and access of meteorological (satellite, observation, forecasts)
\item Ensure \textbf{user requirements} with Product Owner, \textbf{implement} and conduct technical reviews
\item \textbf{Deploy in operation} in several customer sites within few months
\item Stack : Docker, Python, RabbitMQ, Grafana, MongoDB
\end{itemize}

\divider

\cvevent{}{Data Science - IA applied to weather forecast}{}{}
\begin{itemize}
\item \textbf{Research, development and operational} deployment for BMKG (Indonesian Meteorological Service)
\item Improving weather forecasts (Model Output Statistics)
\item Improving \textbf{risk prediction} (Impact Based Forecast)
\item Stack : HPC, Slurm, scikit-learn, Notebook
\end{itemize}

\divider

\cvevent{}{Technical coordinator for Integrated Project}{}{}
\begin{itemize}
\item Coordination for \textbf{large turnkey integrated project}, modernizing National Meteorological Services (NMS) worldwide. Within a multi-team environment, with i\textbf{nternational customers}
\item Architecture design, \textbf{system and data integration}, operation, training and support
\item Key achievements : Indonesia (2022), Angola (2020), ASECNA (2018), Indonesia (2014), India (2010)
\end{itemize}

\divider

\cvevent{}{Product Owner Supervision}{}{}
\begin{itemize}
\item Remote, heterogeneous site supervision worldwide
\item Software, system, hardware and observation sites
\item Limited connectivity, local and centralised supervision
\item Stack : Zabbix, Grafana

\end{itemize}

\medskip
\cvsection{Education}

\cvevent{Computer Science Engineer}{ENSSAT}{Sept. 2005 - 2008}{Lannion, France}
Including ERASMUS at Tempere University of Technology, Finland

\divider

\cvevent{Electronic Industrial Computing}{IUT GEII}{Sept 2003 - 2005}{Cachan, France}
Including intership at Quebec University at Rimouski, Canada


%% Switch to the right column. This will now automatically move to the second
%% page if the content is too long.
\switchcolumn

\cvsection{Strengths}

\textbf{\color{accent}\large{Technical skills}}\par\medskip

\cvskill{Python}{5}
\cvskill{Ansible}{5}
\cvskill{Kubernetes}{3} %% supports X.5 values.

\divider

\textbf{\color{accent}\large{Relational}}\par\medskip

\cvtag{Team work}
\cvtag{Empathetic}
\cvtag{Innovative}
\cvtag{Organized}
\cvtag{Problem-solver}

\divider

\textbf{\color{accent}\large{Stack software}}\par\medskip

\cvtag{Python}
\cvtag{prefect}
\cvtag{FastAPI}
\cvtag{Notebook}
\cvtag{MongoDB}
\cvtag{PostreSQL}

\divider

\textbf{\color{accent}\large{Stack Devops}}\par\medskip
\medskip

\cvtag{Ansible}
\cvtag{Gitlab}
\cvtag{Linux}
\cvtag{VMWare}
\cvtag{Docker}
\cvtag{Kubernetes}

\bigskip
\cvsection{Languages}

\cvskill{French}{5}
\cvskill{English}{4}



\bigskip
\cvsection{About me}

\cvachievement{\faMountain}{Rock Climbing}{}
\cvachievement{\faRunning}{Running}{}
\cvachievement{\faSkiing}{Ski Touring}{}
\cvachievement{\faGlobeEurope}{2012 : Backpacking to Oceania}{}


\end{paracol}

\end{document}
